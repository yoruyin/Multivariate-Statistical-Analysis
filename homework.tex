% Options for packages loaded elsewhere
\PassOptionsToPackage{unicode}{hyperref}
\PassOptionsToPackage{hyphens}{url}
%
\documentclass[
]{article}
\usepackage{lmodern}
\usepackage{amssymb,amsmath}
\usepackage{ifxetex,ifluatex}
\ifnum 0\ifxetex 1\fi\ifluatex 1\fi=0 % if pdftex
  \usepackage[T1]{fontenc}
  \usepackage[utf8]{inputenc}
  \usepackage{textcomp} % provide euro and other symbols
\else % if luatex or xetex
  \usepackage{unicode-math}
  \defaultfontfeatures{Scale=MatchLowercase}
  \defaultfontfeatures[\rmfamily]{Ligatures=TeX,Scale=1}
\fi
% Use upquote if available, for straight quotes in verbatim environments
\IfFileExists{upquote.sty}{\usepackage{upquote}}{}
\IfFileExists{microtype.sty}{% use microtype if available
  \usepackage[]{microtype}
  \UseMicrotypeSet[protrusion]{basicmath} % disable protrusion for tt fonts
}{}
\makeatletter
\@ifundefined{KOMAClassName}{% if non-KOMA class
  \IfFileExists{parskip.sty}{%
    \usepackage{parskip}
  }{% else
    \setlength{\parindent}{0pt}
    \setlength{\parskip}{6pt plus 2pt minus 1pt}}
}{% if KOMA class
  \KOMAoptions{parskip=half}}
\makeatother
\usepackage{xcolor}
\IfFileExists{xurl.sty}{\usepackage{xurl}}{} % add URL line breaks if available
\IfFileExists{bookmark.sty}{\usepackage{bookmark}}{\usepackage{hyperref}}
\hypersetup{
  pdftitle={多元统计分析期末作业},
  pdfauthor={统计1701 尹恒},
  hidelinks,
  pdfcreator={LaTeX via pandoc}}
\urlstyle{same} % disable monospaced font for URLs
\usepackage[margin=1in]{geometry}
\usepackage{color}
\usepackage{fancyvrb}
\newcommand{\VerbBar}{|}
\newcommand{\VERB}{\Verb[commandchars=\\\{\}]}
\DefineVerbatimEnvironment{Highlighting}{Verbatim}{commandchars=\\\{\}}
% Add ',fontsize=\small' for more characters per line
\usepackage{framed}
\definecolor{shadecolor}{RGB}{248,248,248}
\newenvironment{Shaded}{\begin{snugshade}}{\end{snugshade}}
\newcommand{\AlertTok}[1]{\textcolor[rgb]{0.94,0.16,0.16}{#1}}
\newcommand{\AnnotationTok}[1]{\textcolor[rgb]{0.56,0.35,0.01}{\textbf{\textit{#1}}}}
\newcommand{\AttributeTok}[1]{\textcolor[rgb]{0.77,0.63,0.00}{#1}}
\newcommand{\BaseNTok}[1]{\textcolor[rgb]{0.00,0.00,0.81}{#1}}
\newcommand{\BuiltInTok}[1]{#1}
\newcommand{\CharTok}[1]{\textcolor[rgb]{0.31,0.60,0.02}{#1}}
\newcommand{\CommentTok}[1]{\textcolor[rgb]{0.56,0.35,0.01}{\textit{#1}}}
\newcommand{\CommentVarTok}[1]{\textcolor[rgb]{0.56,0.35,0.01}{\textbf{\textit{#1}}}}
\newcommand{\ConstantTok}[1]{\textcolor[rgb]{0.00,0.00,0.00}{#1}}
\newcommand{\ControlFlowTok}[1]{\textcolor[rgb]{0.13,0.29,0.53}{\textbf{#1}}}
\newcommand{\DataTypeTok}[1]{\textcolor[rgb]{0.13,0.29,0.53}{#1}}
\newcommand{\DecValTok}[1]{\textcolor[rgb]{0.00,0.00,0.81}{#1}}
\newcommand{\DocumentationTok}[1]{\textcolor[rgb]{0.56,0.35,0.01}{\textbf{\textit{#1}}}}
\newcommand{\ErrorTok}[1]{\textcolor[rgb]{0.64,0.00,0.00}{\textbf{#1}}}
\newcommand{\ExtensionTok}[1]{#1}
\newcommand{\FloatTok}[1]{\textcolor[rgb]{0.00,0.00,0.81}{#1}}
\newcommand{\FunctionTok}[1]{\textcolor[rgb]{0.00,0.00,0.00}{#1}}
\newcommand{\ImportTok}[1]{#1}
\newcommand{\InformationTok}[1]{\textcolor[rgb]{0.56,0.35,0.01}{\textbf{\textit{#1}}}}
\newcommand{\KeywordTok}[1]{\textcolor[rgb]{0.13,0.29,0.53}{\textbf{#1}}}
\newcommand{\NormalTok}[1]{#1}
\newcommand{\OperatorTok}[1]{\textcolor[rgb]{0.81,0.36,0.00}{\textbf{#1}}}
\newcommand{\OtherTok}[1]{\textcolor[rgb]{0.56,0.35,0.01}{#1}}
\newcommand{\PreprocessorTok}[1]{\textcolor[rgb]{0.56,0.35,0.01}{\textit{#1}}}
\newcommand{\RegionMarkerTok}[1]{#1}
\newcommand{\SpecialCharTok}[1]{\textcolor[rgb]{0.00,0.00,0.00}{#1}}
\newcommand{\SpecialStringTok}[1]{\textcolor[rgb]{0.31,0.60,0.02}{#1}}
\newcommand{\StringTok}[1]{\textcolor[rgb]{0.31,0.60,0.02}{#1}}
\newcommand{\VariableTok}[1]{\textcolor[rgb]{0.00,0.00,0.00}{#1}}
\newcommand{\VerbatimStringTok}[1]{\textcolor[rgb]{0.31,0.60,0.02}{#1}}
\newcommand{\WarningTok}[1]{\textcolor[rgb]{0.56,0.35,0.01}{\textbf{\textit{#1}}}}
\usepackage{longtable,booktabs}
% Correct order of tables after \paragraph or \subparagraph
\usepackage{etoolbox}
\makeatletter
\patchcmd\longtable{\par}{\if@noskipsec\mbox{}\fi\par}{}{}
\makeatother
% Allow footnotes in longtable head/foot
\IfFileExists{footnotehyper.sty}{\usepackage{footnotehyper}}{\usepackage{footnote}}
\makesavenoteenv{longtable}
\usepackage{graphicx,grffile}
\makeatletter
\def\maxwidth{\ifdim\Gin@nat@width>\linewidth\linewidth\else\Gin@nat@width\fi}
\def\maxheight{\ifdim\Gin@nat@height>\textheight\textheight\else\Gin@nat@height\fi}
\makeatother
% Scale images if necessary, so that they will not overflow the page
% margins by default, and it is still possible to overwrite the defaults
% using explicit options in \includegraphics[width, height, ...]{}
\setkeys{Gin}{width=\maxwidth,height=\maxheight,keepaspectratio}
% Set default figure placement to htbp
\makeatletter
\def\fps@figure{htbp}
\makeatother
\setlength{\emergencystretch}{3em} % prevent overfull lines
\providecommand{\tightlist}{%
  \setlength{\itemsep}{0pt}\setlength{\parskip}{0pt}}
\setcounter{secnumdepth}{-\maxdimen} % remove section numbering
%\documentclass{ctexart}
%\usepackage{ctexart}
\usepackage{ctex}

\title{多元统计分析期末作业}
\author{统计1701 尹恒}
\date{2020/5/18}

\begin{document}
\maketitle

\hypertarget{ux7b2cux4e00ux9898}{%
\subsection{第一题}\label{ux7b2cux4e00ux9898}}

\begin{Shaded}
\begin{Highlighting}[]
\NormalTok{data1=}\KeywordTok{read.table}\NormalTok{(}\StringTok{"T1.DAT"}\NormalTok{)   }\CommentTok{#读入数据}
\NormalTok{new_data1<-data1[,}\KeywordTok{c}\NormalTok{(}\DecValTok{1}\OperatorTok{:}\DecValTok{5}\NormalTok{)]    }\CommentTok{#提取前五列作主成分分析}
\KeywordTok{apply}\NormalTok{(new_data1,}\DecValTok{2}\NormalTok{,mean)    }\CommentTok{#计算样本均值向量}
\end{Highlighting}
\end{Shaded}

\begin{verbatim}
##       V1       V2       V3       V4       V5 
## 15.66923 17.07692 18.78462 15.50000 11.73077
\end{verbatim}

\hypertarget{aux786eux5b9aux80fdux6709ux6548ux7684ux7efcux5408ux6837ux672cux53d8ux5f02ux6027ux7684ux6070ux5f53ux7684ux6210ux5206ux4e2aux6570ux6784ux9020ux5d16ux5e95ux788eux77f3ux56feux5e2eux52a9ux4f60ux7684ux6c42ux89e3}{%
\subsubsection{(a)确定能有效的综合样本变异性的恰当的成分个数。构造崖底碎石图帮助你的求解。}\label{aux786eux5b9aux80fdux6709ux6548ux7684ux7efcux5408ux6837ux672cux53d8ux5f02ux6027ux7684ux6070ux5f53ux7684ux6210ux5206ux4e2aux6570ux6784ux9020ux5d16ux5e95ux788eux77f3ux56feux5e2eux52a9ux4f60ux7684ux6c42ux89e3}}

\hypertarget{ux901aux8fc7ux534fux65b9ux5deeux9635sux4f5cux4e3bux6210ux5206ux5206ux6790}{%
\paragraph{通过协方差阵S作主成分分析}\label{ux901aux8fc7ux534fux65b9ux5deeux9635sux4f5cux4e3bux6210ux5206ux5206ux6790}}

\begin{Shaded}
\begin{Highlighting}[]
\KeywordTok{print}\NormalTok{(}\KeywordTok{cov}\NormalTok{(new_data1),}\DataTypeTok{digits =} \DecValTok{4}\NormalTok{) }\CommentTok{#输出协方差矩阵}
\end{Highlighting}
\end{Shaded}

\begin{verbatim}
##         V1      V2       V3      V4      V5
## V1  34.750 -4.2767 -18.0718 -15.973   5.716
## V2  -4.277 17.5134   0.4198  -7.868  -8.723
## V3 -18.072  0.4198  29.8447   9.349 -13.942
## V4 -15.973 -7.8682   9.3488  33.043  -9.942
## V5   5.716 -8.7233 -13.9422  -9.942  26.958
\end{verbatim}

\begin{Shaded}
\begin{Highlighting}[]
\KeywordTok{print}\NormalTok{(}\KeywordTok{eigen}\NormalTok{(}\KeywordTok{cov}\NormalTok{(new_data1))) }\CommentTok{#输出特征值-特征向量}
\end{Highlighting}
\end{Shaded}

\begin{verbatim}
## eigen() decomposition
## $values
## [1] 68.752385 31.508994 23.100973 16.354182  2.392411
## 
## $vectors
##             [,1]        [,2]       [,3]        [,4]       [,5]
## [1,]  0.57943538  0.07917988 -0.6428795 -0.30939267 -0.3859629
## [2,] -0.04165689  0.61192825  0.1399143  0.51462195 -0.5825777
## [3,] -0.52428496  0.21883511  0.1192554 -0.73403767 -0.3524249
## [4,] -0.49309245 -0.57215650 -0.4221873  0.30427403 -0.3983365
## [5,]  0.38013742 -0.49398633  0.6120997 -0.08970196 -0.4782893
\end{verbatim}

\begin{Shaded}
\begin{Highlighting}[]
\NormalTok{res1=}\KeywordTok{princomp}\NormalTok{(new_data1,}\DataTypeTok{cor=}\OtherTok{FALSE}\NormalTok{) }\CommentTok{#当cor=FALSE表示用样本的协方差阵S做主成分分析}
\KeywordTok{summary}\NormalTok{(res1,}\DataTypeTok{loadings=}\OtherTok{TRUE}\NormalTok{)}
\end{Highlighting}
\end{Shaded}

\begin{verbatim}
## Importance of components:
##                           Comp.1    Comp.2    Comp.3   Comp.4     Comp.5
## Standard deviation     8.2597530 5.5916560 4.7878255 4.028446 1.54078160
## Proportion of Variance 0.4838005 0.2217242 0.1625582 0.115082 0.01683505
## Cumulative Proportion  0.4838005 0.7055248 0.8680829 0.983165 1.00000000
## 
## Loadings:
##    Comp.1 Comp.2 Comp.3 Comp.4 Comp.5
## V1  0.579         0.643  0.309  0.386
## V2         0.612 -0.140 -0.515  0.583
## V3 -0.524  0.219 -0.119  0.734  0.352
## V4 -0.493 -0.572  0.422 -0.304  0.398
## V5  0.380 -0.494 -0.612         0.478
\end{verbatim}

\begin{Shaded}
\begin{Highlighting}[]
\KeywordTok{screeplot}\NormalTok{(res1,}\DataTypeTok{type=}\StringTok{"lines"}\NormalTok{)}\CommentTok{#输出崖底碎石图}
\end{Highlighting}
\end{Shaded}

\includegraphics{homework_files/figure-latex/unnamed-chunk-2-1.pdf}
由上述程序整理可得主成分的系数为:

\begin{longtable}[]{@{}cccccc@{}}
\toprule
变量 & \(\hat{e}_{1}\) & \(\hat{e}_{2}\) & \(\hat{e}_{3}\) & \(\hat{e}_{4}\) &
\(\hat{e}_{5}\)\tabularnewline
\midrule
\endhead
独立性 & 0.579 & - & 0.643 & 0.309 & 0.386\tabularnewline
支持力 & - & 0.612 & -0.140 & -0.515 & 0.583\tabularnewline
仁爱心 & -0.524 & 0.219 & -0.119 & 0.734 & 0.352\tabularnewline
顺从性 & -0.493 & -0.572 & 0.422 & -0.304 & 0.398\tabularnewline
领导能力 & 0.380 & -0.494 & -0.612 & - & 0.478\tabularnewline
方差($\hat{\lambda} $) & 68.752 & 31.509 & 23.101 & 16.354 &
2.392\tabularnewline
占总方差的累计百分比 & 0.484 & 0.706 & 0.868 & 0.983 &
1.000\tabularnewline
\bottomrule
\end{longtable}

由碎石图可以看出选择前三个主成分较为合理,由表格得前三个主成分占总方差的累计百分比为86.8\%,与碎石图得到的结果基本一致。

\hypertarget{ux901aux8fc7ux76f8ux5173ux9635rux4f5cux4e3bux6210ux5206ux5206ux6790}{%
\paragraph{通过相关阵R作主成分分析}\label{ux901aux8fc7ux76f8ux5173ux9635rux4f5cux4e3bux6210ux5206ux5206ux6790}}

\begin{Shaded}
\begin{Highlighting}[]
\KeywordTok{print}\NormalTok{(}\KeywordTok{cor}\NormalTok{(new_data1),}\DataTypeTok{digits =} \DecValTok{4}\NormalTok{) }\CommentTok{#输出相关矩阵}
\end{Highlighting}
\end{Shaded}

\begin{verbatim}
##         V1       V2       V3      V4      V5
## V1  1.0000 -0.17336 -0.56116 -0.4714  0.1868
## V2 -0.1734  1.00000  0.01836 -0.3271 -0.4015
## V3 -0.5612  0.01836  1.00000  0.2977 -0.4915
## V4 -0.4714 -0.32708  0.29771  1.0000 -0.3331
## V5  0.1868 -0.40147 -0.49153 -0.3331  1.0000
\end{verbatim}

\begin{Shaded}
\begin{Highlighting}[]
\KeywordTok{print}\NormalTok{(}\KeywordTok{eigen}\NormalTok{(}\KeywordTok{cor}\NormalTok{(new_data1))) }\CommentTok{#输出特征值-特征向量}
\end{Highlighting}
\end{Shaded}

\begin{verbatim}
## eigen() decomposition
## $values
## [1] 2.19662443 1.36824960 0.75586304 0.58878599 0.09047694
## 
## $vectors
##            [,1]         [,2]       [,3]         [,4]      [,5]
## [1,]  0.5209626  0.086521361  0.6674512  0.253099293 0.4599582
## [2,] -0.1213677  0.788216689 -0.1870605 -0.350892684 0.4537257
## [3,] -0.5482732 -0.007941356 -0.1150943  0.732694760 0.3863226
## [4,] -0.4391410 -0.490952547  0.2949415 -0.525281896 0.4507873
## [5,]  0.4694885 -0.360736798 -0.6475184 -0.007238184 0.4797052
\end{verbatim}

\begin{Shaded}
\begin{Highlighting}[]
\NormalTok{res2=}\KeywordTok{princomp}\NormalTok{(new_data1,}\DataTypeTok{cor=}\OtherTok{TRUE}\NormalTok{) }\CommentTok{#当cor=TRUE表示用样本的相关矩阵R做主成分分析}
\KeywordTok{summary}\NormalTok{(res2,}\DataTypeTok{loadings=}\OtherTok{TRUE}\NormalTok{)}
\end{Highlighting}
\end{Shaded}

\begin{verbatim}
## Importance of components:
##                           Comp.1    Comp.2    Comp.3    Comp.4     Comp.5
## Standard deviation     1.4821014 1.1697220 0.8694038 0.7673239 0.30079386
## Proportion of Variance 0.4393249 0.2736499 0.1511726 0.1177572 0.01809539
## Cumulative Proportion  0.4393249 0.7129748 0.8641474 0.9819046 1.00000000
## 
## Loadings:
##    Comp.1 Comp.2 Comp.3 Comp.4 Comp.5
## V1  0.521         0.667  0.253  0.460
## V2 -0.121  0.788 -0.187 -0.351  0.454
## V3 -0.548        -0.115  0.733  0.386
## V4 -0.439 -0.491  0.295 -0.525  0.451
## V5  0.469 -0.361 -0.648         0.480
\end{verbatim}

\begin{Shaded}
\begin{Highlighting}[]
\KeywordTok{screeplot}\NormalTok{(res2,}\DataTypeTok{type=}\StringTok{"lines"}\NormalTok{) }\CommentTok{#输出崖底碎石图}
\end{Highlighting}
\end{Shaded}

\includegraphics{homework_files/figure-latex/unnamed-chunk-3-1.pdf}
由上述程序整理可得主成分的系数为:

\begin{longtable}[]{@{}cccccc@{}}
\toprule
变量 & \(\hat{e}_{1}\) & \(\hat{e}_{2}\) & \(\hat{e}_{3}\) & \(\hat{e}_{4}\) &
\(\hat{e}_{5}\)\tabularnewline
\midrule
\endhead
独立性 & 0.521 & - & 0.667 & 0.253 & 0.460\tabularnewline
支持力 & -0.121 & 0.788 & -0.187 & -0.351 & 0.454\tabularnewline
仁爱心 & -0.548 & - & -0.115 & 0.733 & 0.386\tabularnewline
顺从性 & -0.439 & -0.491 & 0.295 & -0.525 & 0.451\tabularnewline
领导能力 & 0.469 & -0.361 & -0.648 & - & 0.480\tabularnewline
方差($\hat{\lambda}$) & 2.197 & 1.368 & 0.756 & 0.589 &
0.090\tabularnewline
占总方差的累计百分比 & 0.439 & 0.713 & 0.864 & 0.981 &
1.000\tabularnewline
\bottomrule
\end{longtable}

由碎石图可以看出选择前三个主成分较为合理,由表格得前三个主成分占总方差的累计百分比为86.4\%,与碎石图得到的结果基本一致。

\hypertarget{bux89e3ux91caux6837ux672cux4e3bux6210ux5206.}{%
\subsubsection{(b)解释样本主成分.}\label{bux89e3ux91caux6837ux672cux4e3bux6210ux5206.}}

选取的三个样本主成分为:
$$ 
y_{1}=0.579\times x_{1}-0.524\times x_{3}-0.493\times x_{4}+0.380\times x_{5}
$$
$$
y_{2}=0.612\times x_{2}+0.219\times x_{3}-0.572\times x_{4}-0.494\times x_{5}
$$
$$
y_{3}=0.643\times x_{1}-0.140\times x_{2}-0.119\times x_{3}+0.422\times x_{4}-0.612\times x_{5}
$$
第一主成分占总方差的48.4\%,其中和独立性与领导能力正相关,和仁爱心与顺从性负相关,表现出秘鲁青年心理大部分呈现独立自主。

第二主成分占总方差的22.2\%,其中和支持力与仁爱心正相关,和领导能力与顺从性负相关,表现出秘鲁青年心理少部分呈现顺从仁爱
。

第三主成分占总方差的16.3\%,是五个因素的线性组合,表现更一般的情况,各种因素都会影响秘鲁青年心理状况。

\hypertarget{cux7528ux524dux4e24ux4e2aux4e3bux6210ux5206ux7684ux503cux5c06y_1y_2ux7684ux503cux753bux5728ux56feux4e2d}{%
\subsubsection{\texorpdfstring{(c)用前两个主成分的值,将$(\hat{y_{1}},\hat{y_{2}})$的值画在图中}{(c)用前两个主成分的值,将(y\_\{1\},y\_\{2\})的值画在图中}}\label{cux7528ux524dux4e24ux4e2aux4e3bux6210ux5206ux7684ux503cux5c06y_1y_2ux7684ux503cux753bux5728ux56feux4e2d}}

\begin{Shaded}
\begin{Highlighting}[]
\KeywordTok{library}\NormalTok{(ggplot2)}
\KeywordTok{attach}\NormalTok{(data1)}\CommentTok{#把第一二主成分的值加入原表格}
\NormalTok{data1}\OperatorTok{$}\NormalTok{Y1<-}\FloatTok{0.579}\OperatorTok{*}\NormalTok{V1}\FloatTok{-0.524}\OperatorTok{*}\NormalTok{V3}\FloatTok{-0.493}\OperatorTok{*}\NormalTok{V4}\FloatTok{+0.380}\OperatorTok{*}\NormalTok{V5 }\CommentTok{#第一主成分}
\NormalTok{data1}\OperatorTok{$}\NormalTok{Y2<-}\FloatTok{0.612}\OperatorTok{*}\NormalTok{V2}\FloatTok{+0.219}\OperatorTok{*}\NormalTok{V3}\FloatTok{-0.572}\OperatorTok{*}\NormalTok{V4}\FloatTok{-0.494}\OperatorTok{*}\NormalTok{V5 }\CommentTok{#第二主成分}
\KeywordTok{detach}\NormalTok{(data1)}
\end{Highlighting}
\end{Shaded}

以性别作为分类指标作图

\begin{Shaded}
\begin{Highlighting}[]
\KeywordTok{ggplot}\NormalTok{(data1,}\KeywordTok{aes}\NormalTok{(}\DataTypeTok{x=}\NormalTok{Y1,}\DataTypeTok{y=}\NormalTok{Y2,}\DataTypeTok{colour=}\NormalTok{V6))}\OperatorTok{+}\KeywordTok{geom_point}\NormalTok{()}
\end{Highlighting}
\end{Shaded}

\includegraphics{homework_files/figure-latex/unnamed-chunk-5-1.pdf}

以地位作为分类指标作图

\begin{Shaded}
\begin{Highlighting}[]
\KeywordTok{ggplot}\NormalTok{(data1,}\KeywordTok{aes}\NormalTok{(}\DataTypeTok{x=}\NormalTok{Y1,}\DataTypeTok{y=}\NormalTok{Y2,}\DataTypeTok{colour=}\NormalTok{V7))}\OperatorTok{+}\KeywordTok{geom_point}\NormalTok{()}
\end{Highlighting}
\end{Shaded}

\includegraphics{homework_files/figure-latex/unnamed-chunk-6-1.pdf}

从图看出在右下角存在几个离群值。

\hypertarget{dux7528ux7b2cux4e00ux4e3bux6210ux5206q-qux56feux89e3ux91caux8be5ux56fe}{%
\subsubsection{(d)用第一主成分Q-Q图,解释该图。}\label{dux7528ux7b2cux4e00ux4e3bux6210ux5206q-qux56feux89e3ux91caux8be5ux56fe}}

\begin{Shaded}
\begin{Highlighting}[]
\NormalTok{lamda<-}\KeywordTok{eigen}\NormalTok{(}\KeywordTok{cov}\NormalTok{(new_data1)) }
\NormalTok{spc_mat<-lamda}\OperatorTok{$}\NormalTok{vectors[,}\DecValTok{1}\NormalTok{]}
\NormalTok{prin_y<-}\KeywordTok{t}\NormalTok{(}\KeywordTok{t}\NormalTok{(spc_mat)}\OperatorTok\KeywordTok{t}\NormalTok{(new_data1))}
\KeywordTok{qqnorm}\NormalTok{(prin_y)}
\KeywordTok{qqline}\NormalTok{(prin_y)}
\end{Highlighting}
\end{Shaded}

\includegraphics{homework_files/figure-latex/unnamed-chunk-7-1.pdf}

此图说明第一主成分在最左端拟合程度不好,并且在最右上角存在可疑点。

\hypertarget{ux7b2cux4e8cux9898}{%
\subsection{第二题}\label{ux7b2cux4e8cux9898}}

\hypertarget{aux6c42ux4e3bux6210ux5206ux89e3ux6216ux6781ux5927ux4f3cux7136ux89e3}{%
\subsubsection{(a)求主成分解或极大似然解}\label{aux6c42ux4e3bux6210ux5206ux89e3ux6216ux6781ux5927ux4f3cux7136ux89e3}}

因子分析,利用psych包中的fa(r=cor2,nfactors=2,fm=``pa'',rotate=``none'')函数,该函数为多元统计分析的一个包;nfactors为因子个数,fm为估计解的方法:pa为主成分法,ml为极大似然估计法;rotate为是否进行旋转

\begin{Shaded}
\begin{Highlighting}[]
\KeywordTok{library}\NormalTok{(psych)}
\end{Highlighting}
\end{Shaded}

\begin{Shaded}
\begin{Highlighting}[]
\NormalTok{data2=}\KeywordTok{read.table}\NormalTok{(}\StringTok{"T2.DAT"}\NormalTok{)}
\NormalTok{cor2<-}\KeywordTok{cor}\NormalTok{(}\KeywordTok{scale}\NormalTok{(data2)) }
\CommentTok{#主成分法}
\NormalTok{m2<-}\KeywordTok{fa}\NormalTok{(}\DataTypeTok{r=}\NormalTok{cor2,}\DataTypeTok{nfactors=}\DecValTok{2}\NormalTok{,}\DataTypeTok{fm=}\StringTok{"pa"}\NormalTok{,}\DataTypeTok{rotate=}\StringTok{"none"}\NormalTok{)}
\end{Highlighting}
\end{Shaded}

\begin{verbatim}
## maximum iteration exceeded
\end{verbatim}

\begin{verbatim}
## Warning in fa.stats(r = r, f = f, phi = phi, n.obs = n.obs, np.obs = np.obs, :
## The estimated weights for the factor scores are probably incorrect. Try a
## different factor score estimation method.
\end{verbatim}

\begin{verbatim}
## Warning in fac(r = r, nfactors = nfactors, n.obs = n.obs, rotate = rotate, : An
## ultra-Heywood case was detected. Examine the results carefully
\end{verbatim}

\begin{Shaded}
\begin{Highlighting}[]
\NormalTok{m2}\OperatorTok{$}\NormalTok{loadings}
\end{Highlighting}
\end{Shaded}

\begin{verbatim}
## 
## Loadings:
##    PA1    PA2   
## V1  0.984 -0.168
## V2  0.933 -0.117
## V3  0.934       
## V4  0.717  0.870
## V5  0.722       
## V6  0.579 -0.291
## V7  0.906 -0.273
## 
##                  PA1   PA2
## SS loadings    4.901 0.965
## Proportion Var 0.700 0.138
## Cumulative Var 0.700 0.838
\end{verbatim}

\begin{Shaded}
\begin{Highlighting}[]
\NormalTok{m3<-}\KeywordTok{fa}\NormalTok{(}\DataTypeTok{r=}\NormalTok{cor2,}\DataTypeTok{nfactors=}\DecValTok{3}\NormalTok{,}\DataTypeTok{fm=}\StringTok{"pa"}\NormalTok{,}\DataTypeTok{rotate=}\StringTok{"none"}\NormalTok{)}
\end{Highlighting}
\end{Shaded}

\begin{verbatim}
## Warning in fa.stats(r = r, f = f, phi = phi, n.obs = n.obs, np.obs = np.obs, :
## The estimated weights for the factor scores are probably incorrect. Try a
## different factor score estimation method.

## Warning in fa.stats(r = r, f = f, phi = phi, n.obs = n.obs, np.obs = np.obs, :
## An ultra-Heywood case was detected. Examine the results carefully
\end{verbatim}

\begin{Shaded}
\begin{Highlighting}[]
\NormalTok{m3}\OperatorTok{$}\NormalTok{loadings}
\end{Highlighting}
\end{Shaded}

\begin{verbatim}
## 
## Loadings:
##    PA1    PA2    PA3   
## V1  0.976              
## V2  0.951        -0.312
## V3  0.936         0.146
## V4  0.652  0.641  0.279
## V5  0.723  0.179       
## V6  0.641 -0.573  0.390
## V7  0.917 -0.175 -0.274
## 
##                  PA1   PA2   PA3
## SS loadings    4.934 0.813 0.424
## Proportion Var 0.705 0.116 0.061
## Cumulative Var 0.705 0.821 0.882
\end{verbatim}

\begin{Shaded}
\begin{Highlighting}[]
\CommentTok{#极大似然估计法}
\NormalTok{m22<-}\KeywordTok{fa}\NormalTok{(}\DataTypeTok{r=}\NormalTok{cor2,}\DataTypeTok{nfactors=}\DecValTok{2}\NormalTok{,}\DataTypeTok{fm=}\StringTok{"ml"}\NormalTok{,}\DataTypeTok{rotate=}\StringTok{"none"}\NormalTok{)}
\NormalTok{m22}\OperatorTok{$}\NormalTok{loadings}
\end{Highlighting}
\end{Shaded}

\begin{verbatim}
## 
## Loadings:
##    ML1    ML2   
## V1  0.695  0.669
## V2  0.669  0.695
## V3  0.795  0.494
## V4  0.983 -0.167
## V5  0.655  0.312
## V6  0.250  0.569
## V7  0.558  0.812
## 
##                  ML1   ML2
## SS loadings    3.333 2.283
## Proportion Var 0.476 0.326
## Cumulative Var 0.476 0.802
\end{verbatim}

\begin{Shaded}
\begin{Highlighting}[]
\NormalTok{m33<-}\KeywordTok{fa}\NormalTok{(}\DataTypeTok{r=}\NormalTok{cor2,}\DataTypeTok{nfactors=}\DecValTok{3}\NormalTok{,}\DataTypeTok{fm=}\StringTok{"ml"}\NormalTok{,}\DataTypeTok{rotate=}\StringTok{"none"}\NormalTok{)}
\NormalTok{m33}\OperatorTok{$}\NormalTok{loadings}
\end{Highlighting}
\end{Shaded}

\begin{verbatim}
## 
## Loadings:
##    ML1    ML3    ML2   
## V1  0.901  0.381       
## V2  0.775  0.600       
## V3  0.931  0.202       
## V4  0.733 -0.118  0.666
## V5  0.689  0.225  0.169
## V6  0.757 -0.132 -0.636
## V7  0.762  0.608 -0.110
## 
##                  ML1   ML3   ML2
## SS loadings    4.445 0.998 0.901
## Proportion Var 0.635 0.143 0.129
## Cumulative Var 0.635 0.778 0.906
\end{verbatim}

\hypertarget{bux6c42ux65cbux8f6cux8f7dux8377ux6bd4ux8f83ux8fd9ux4e24ux7ec4ux65cbux8f6cux8f7dux8377ux89e3ux91caux56e0ux5b50ux89e3}{%
\subsubsection{(b)求旋转载荷,比较这两组旋转载荷,解释因子解}\label{bux6c42ux65cbux8f6cux8f7dux8377ux6bd4ux8f83ux8fd9ux4e24ux7ec4ux65cbux8f6cux8f7dux8377ux89e3ux91caux56e0ux5b50ux89e3}}

\begin{Shaded}
\begin{Highlighting}[]
\NormalTok{m20<-}\KeywordTok{fa}\NormalTok{(}\DataTypeTok{r=}\NormalTok{cor2,}\DataTypeTok{nfactors=}\DecValTok{2}\NormalTok{,}\DataTypeTok{fm=}\StringTok{"ml"}\NormalTok{,}\DataTypeTok{rotate=}\StringTok{"varimax"}\NormalTok{)}
\NormalTok{m20}\OperatorTok{$}\NormalTok{loadings}
\end{Highlighting}
\end{Shaded}

\begin{verbatim}
## 
## Loadings:
##    ML2   ML1  
## V1 0.852 0.452
## V2 0.868 0.419
## V3 0.717 0.602
## V4 0.148 0.987
## V5 0.501 0.525
## V6 0.619      
## V7 0.946 0.277
## 
##                  ML2   ML1
## SS loadings    3.545 2.071
## Proportion Var 0.506 0.296
## Cumulative Var 0.506 0.802
\end{verbatim}

第一因子可以把数学能力与销售利润联系起来,表现销售人员的销售能力
第二因子可以把创造力和新客户销售额和联系起来,表现销售人员推销能力

\begin{Shaded}
\begin{Highlighting}[]
\NormalTok{m30<-}\KeywordTok{fa}\NormalTok{(}\DataTypeTok{r=}\NormalTok{cor2,}\DataTypeTok{nfactors=}\DecValTok{3}\NormalTok{,}\DataTypeTok{fm=}\StringTok{"ml"}\NormalTok{,}\DataTypeTok{rotate=}\StringTok{"varimax"}\NormalTok{)}
\NormalTok{m30}\OperatorTok{$}\NormalTok{loadings}
\end{Highlighting}
\end{Shaded}

\begin{verbatim}
## 
## Loadings:
##    ML3   ML1   ML2  
## V1 0.793 0.374 0.438
## V2 0.911 0.317 0.185
## V3 0.651 0.544 0.438
## V4 0.255 0.964      
## V5 0.542 0.465 0.207
## V6 0.299       0.950
## V7 0.917 0.180 0.298
## 
##                  ML3   ML1   ML2
## SS loadings    3.175 1.718 1.453
## Proportion Var 0.454 0.245 0.208
## Cumulative Var 0.454 0.699 0.906
\end{verbatim}

第一,二因子同上。
第三因子把抽象推理能力和新客户销售额和销售增长联系起来,表现销售人员的判断能力。
比较这两组旋转载荷,三个因子的累计方差达到了90\%,比两个因子的累计方差高。

\hypertarget{cux5217ux51faux5171ux6027ux65b9ux5deeux7279ux6b8aux65b9ux5deeux6bd4ux8f83ux7ed3ux679cux5e76ux89e3ux91ca}{%
\subsubsection{(c)列出共性方差,特殊方差,比较结果并解释}\label{cux5217ux51faux5171ux6027ux65b9ux5deeux7279ux6b8aux65b9ux5deeux6bd4ux8f83ux7ed3ux679cux5e76ux89e3ux91ca}}

列出分析的完整数据如下:(communalities为共性方差,特殊方差为1-共性方差)

\begin{Shaded}
\begin{Highlighting}[]
\NormalTok{m20}\OperatorTok{$}\NormalTok{communalities}
\end{Highlighting}
\end{Shaded}

\begin{verbatim}
##        V1        V2        V3        V4        V5        V6        V7 
## 0.9308084 0.9296171 0.8766888 0.9950000 0.5264156 0.3863585 0.9711829
\end{verbatim}

\begin{Shaded}
\begin{Highlighting}[]
\NormalTok{m30}\OperatorTok{$}\NormalTok{communalities}
\end{Highlighting}
\end{Shaded}

\begin{verbatim}
##        V1        V2        V3        V4        V5        V6        V7 
## 0.9614288 0.9655182 0.9118758 0.9950000 0.5533880 0.9950000 0.9624919
\end{verbatim}

两个因子的共性方差和特殊方差为:

\begin{longtable}[]{@{}ccc@{}}
\toprule
& 共性方差 & 特殊方差\tabularnewline
\midrule
\endhead
销售增长 & 0.93 & 0.069\tabularnewline
销售利润 & 0.93 & 0.070\tabularnewline
新客户销售额 & 0.88 & 0.123\tabularnewline
创造力 & 1.00 & 0.005\tabularnewline
机械推理 & 0.53 & 0.474\tabularnewline
抽象推理 & 0.39 & 0.614\tabularnewline
数学能力 & 0.97 & 0.029\tabularnewline
\bottomrule
\end{longtable}

三个因子的共性方差和特殊方差为:

\begin{longtable}[]{@{}ccc@{}}
\toprule
& 共性方差 & 特殊方差\tabularnewline
\midrule
\endhead
销售增长 & 0.96 & 0.039\tabularnewline
销售利润 & 0.97 & 0.034\tabularnewline
新客户销售额 & 0.91 & 0.088\tabularnewline
创造力 & 1.00 & 0.005\tabularnewline
机械推理 & 0.55 & 0.447\tabularnewline
抽象推理 & 1.00 & 0.005\tabularnewline
数学能力 & 0.96 & 0.038\tabularnewline
\bottomrule
\end{longtable}

比较两个表格,三个因子的共性方差基本都接近1,并且三个因子的累计方差高,所以选三个因子

\hypertarget{dux5bf9m2ux548cm3ux505aux5047ux8bbeux68c0ux9a8c}{%
\subsubsection{\texorpdfstring{(d)对\(m=2\)和\(m=3\)做假设检验。}{(d)对m=2和m=3做假设检验。}}\label{dux5bf9m2ux548cm3ux505aux5047ux8bbeux68c0ux9a8c}}

由公式(9-39)如下,把n=50,p=7,m=2,3代入得:

\[(n-1-(2 p+4 m+5) / 6) \ln \frac{\left|\hat{\mathbf{L}} \hat{\mathbf{L}}^{\prime}+\hat{\Psi}\right|}{\left|\mathbf{S}_{n}\right|}>\chi^{2}_{\left[(p-m)^{2}-p-m\right] / 2}(\alpha)\]

\[43.833\times ln(\frac{0.000075933}{0.000018427})=62.1>\chi^{2}(0.01)=11.3\]

所以我们拒绝原假设\(H_{0}\),综合以上分析选择\(m=3\)。

\hypertarget{ux7b2cux4e09ux9898}{%
\subsection{第三题}\label{ux7b2cux4e09ux9898}}

\hypertarget{aux4f7fux7528ux4e8cux6b21ux5224ux522bux65b9ux6cd5ux5c06x_03.51.75ux5206ux7c7bux5230ux603bux4f53pi_1pi_2ux6216pi_3}{%
\subsubsection{\texorpdfstring{(a)使用二次判别方法将\(X_{0}^{'}=[3.5,1.75]\)分类到总体\(\pi_{1},\pi_{2}或\pi_{3}\)}{(a)使用二次判别方法将X\_\{0\}\^{}\{'\}={[}3.5,1.75{]}分类到总体\textbackslash pi\_\{1\},\textbackslash pi\_\{2\}或\textbackslash pi\_\{3\}}}\label{aux4f7fux7528ux4e8cux6b21ux5224ux522bux65b9ux6cd5ux5c06x_03.51.75ux5206ux7c7bux5230ux603bux4f53pi_1pi_2ux6216pi_3}}

\begin{Shaded}
\begin{Highlighting}[]
\KeywordTok{library}\NormalTok{(MASS)}
\NormalTok{data3=}\KeywordTok{read.table}\NormalTok{(}\StringTok{"T3.DAT"}\NormalTok{)}
\NormalTok{qd<-}\KeywordTok{qda}\NormalTok{(V5}\OperatorTok{~}\NormalTok{V2}\OperatorTok{+}\NormalTok{V4,data3,}\DataTypeTok{prior=}\KeywordTok{c}\NormalTok{(}\DecValTok{1}\OperatorTok{/}\DecValTok{3}\NormalTok{,}\DecValTok{1}\OperatorTok{/}\DecValTok{3}\NormalTok{,}\DecValTok{1}\OperatorTok{/}\DecValTok{3}\NormalTok{)) }\CommentTok{#二次判别}
\end{Highlighting}
\end{Shaded}

\begin{Shaded}
\begin{Highlighting}[]
\KeywordTok{predict}\NormalTok{(qd,}\DataTypeTok{newdata =} \KeywordTok{data.frame}\NormalTok{(}\DataTypeTok{V2=}\FloatTok{3.5}\NormalTok{,}\DataTypeTok{V4=}\FloatTok{1.75}\NormalTok{))}
\end{Highlighting}
\end{Shaded}

\begin{verbatim}
## $class
## [1] 2
## Levels: 1 2 3
## 
## $posterior
##              1         2         3
## 1 6.391308e-46 0.7807453 0.2192547
\end{verbatim}

根据后验概率第二类最大,所以分类在第二类

\hypertarget{bux4f7fux7528ux7ebfux6027ux5224ux522bux65b9ux6cd5ux5c06x_03.51.75ux5206ux7c7bux5230ux603bux4f53pi_1pi_2ux6216pi_3}{%
\subsubsection{\texorpdfstring{(b)使用线性判别方法将\(X_{0}^{'}=[3.5,1.75]\)分类到总体\(\pi_{1},\pi_{2}或\pi_{3}\)}{(b)使用线性判别方法将X\_\{0\}\^{}\{'\}={[}3.5,1.75{]}分类到总体\textbackslash pi\_\{1\},\textbackslash pi\_\{2\}或\textbackslash pi\_\{3\}}}\label{bux4f7fux7528ux7ebfux6027ux5224ux522bux65b9ux6cd5ux5c06x_03.51.75ux5206ux7c7bux5230ux603bux4f53pi_1pi_2ux6216pi_3}}

\begin{Shaded}
\begin{Highlighting}[]
\NormalTok{ld<-}\KeywordTok{lda}\NormalTok{(V5}\OperatorTok{~}\NormalTok{V2}\OperatorTok{+}\NormalTok{V4,data3,}\DataTypeTok{prior=}\KeywordTok{c}\NormalTok{(}\DecValTok{1}\OperatorTok{/}\DecValTok{3}\NormalTok{,}\DecValTok{1}\OperatorTok{/}\DecValTok{3}\NormalTok{,}\DecValTok{1}\OperatorTok{/}\DecValTok{3}\NormalTok{)) }\CommentTok{#线性判别}
\end{Highlighting}
\end{Shaded}

\begin{Shaded}
\begin{Highlighting}[]
\KeywordTok{predict}\NormalTok{(ld,}\DataTypeTok{newdata =} \KeywordTok{data.frame}\NormalTok{(}\DataTypeTok{V2=}\FloatTok{3.5}\NormalTok{,}\DataTypeTok{V4=}\FloatTok{1.75}\NormalTok{))}
\end{Highlighting}
\end{Shaded}

\begin{verbatim}
## $class
## [1] 2
## Levels: 1 2 3
## 
## $posterior
##              1         2         3
## 1 3.209389e-14 0.7187594 0.2812406
## 
## $x
##        LD1      LD2
## 1 2.136514 1.636255
\end{verbatim}

所以线性判别得分为2.136514和1.636255

根据后验概率第二类最大,所以分类在第二类

\hypertarget{cux7528bux4e2dux7684ux7ebfux6027ux5224ux522bux51fdux6570ux5c06ux6837ux672cux89c2ux6d4bux503cux5206ux7c7bux8ba1ux7b97aperux548chateaer}{%
\subsubsection{\texorpdfstring{(c)用(b)中的线性判别函数将样本观测值分类。计算\(APER\)和\(\hat{E}(AER)\)}{(c)用(b)中的线性判别函数将样本观测值分类。计算APER和\textbackslash hat\{E\}(AER)}}\label{cux7528bux4e2dux7684ux7ebfux6027ux5224ux522bux51fdux6570ux5c06ux6837ux672cux89c2ux6d4bux503cux5206ux7c7bux8ba1ux7b97aperux548chateaer}}

\begin{Shaded}
\begin{Highlighting}[]
\NormalTok{pred<-}\KeywordTok{predict}\NormalTok{(ld) }\CommentTok{#用模型对学习样本分类}
\NormalTok{tab1<-}\KeywordTok{table}\NormalTok{(data3}\OperatorTok{$}\NormalTok{V5,pred}\OperatorTok{$}\NormalTok{class)}
\NormalTok{tab1}
\end{Highlighting}
\end{Shaded}

\begin{verbatim}
##    
##      1  2  3
##   1 50  0  0
##   2  0 49  1
##   3  0  4 46
\end{verbatim}

观察表格其中出错了5个值,根据公式\(APER=\frac{5}{150}=0.033\)

\begin{Shaded}
\begin{Highlighting}[]
\NormalTok{ld1<-}\KeywordTok{lda}\NormalTok{(V5}\OperatorTok{~}\NormalTok{V2}\OperatorTok{+}\NormalTok{V4,data3,}\DataTypeTok{prior=}\KeywordTok{c}\NormalTok{(}\DecValTok{1}\OperatorTok{/}\DecValTok{3}\NormalTok{,}\DecValTok{1}\OperatorTok{/}\DecValTok{3}\NormalTok{,}\DecValTok{1}\OperatorTok{/}\DecValTok{3}\NormalTok{),}\DataTypeTok{CV=}\NormalTok{T) }\CommentTok{#CV=T 运用提留方法}
\NormalTok{tab2<-}\KeywordTok{table}\NormalTok{(data3}\OperatorTok{$}\NormalTok{V5,ld1}\OperatorTok{$}\NormalTok{class)}
\NormalTok{tab2}
\end{Highlighting}
\end{Shaded}

\begin{verbatim}
##    
##      1  2  3
##   1 50  0  0
##   2  0 48  2
##   3  0  4 46
\end{verbatim}

观察表格其中出错了6个值根据公式\(\hat{E}(AER)=\frac{6}{150}=0.04\)

\hypertarget{ux7b2cux56dbux9898}{%
\subsection{第四题}\label{ux7b2cux56dbux9898}}

\hypertarget{aux7528ux5355ux8fdeux63a5ux6cd5ux5b8cux5168ux8fdeux63a5ux6cd5ux5bf9ux519cux573aux505aux805aux7c7bux6784ux9020ux8fdeux63a5ux6811ux56feux5e76ux6bd4ux8f83ux7ed3ux679c}{%
\subsubsection{(a)用单连接法,完全连接法对农场做聚类。构造连接树图并比较结果。}\label{aux7528ux5355ux8fdeux63a5ux6cd5ux5b8cux5168ux8fdeux63a5ux6cd5ux5bf9ux519cux573aux505aux805aux7c7bux6784ux9020ux8fdeux63a5ux6811ux56feux5e76ux6bd4ux8f83ux7ed3ux679c}}

\begin{Shaded}
\begin{Highlighting}[]
\NormalTok{data4=}\KeywordTok{read.table}\NormalTok{(}\StringTok{"T4.DAT"}\NormalTok{)   }\CommentTok{#读入数据}
\NormalTok{data4<-data4[}\OperatorTok{-}\KeywordTok{c}\NormalTok{(}\DecValTok{25}\NormalTok{,}\DecValTok{34}\NormalTok{,}\DecValTok{69}\NormalTok{,}\DecValTok{72}\NormalTok{),]   }\CommentTok{#去掉离群值25,34,69,72}
\KeywordTok{dim}\NormalTok{(data4)   }\CommentTok{#输出数据的维度}
\end{Highlighting}
\end{Shaded}

\begin{verbatim}
## [1] 72  9
\end{verbatim}

\begin{Shaded}
\begin{Highlighting}[]
\NormalTok{dist4=}\KeywordTok{dist}\NormalTok{(}\KeywordTok{scale}\NormalTok{(data4), }\DataTypeTok{method =} \StringTok{"euclidean"}\NormalTok{, }\DataTypeTok{p =} \DecValTok{2}\NormalTok{)}
\CommentTok{#对标准化后的数据计算欧氏距离,"euclidean"表示欧氏距离,维度为2}
\NormalTok{D4_single<-}\KeywordTok{hclust}\NormalTok{(dist4,}\DataTypeTok{method=}\StringTok{"single"}\NormalTok{)}
\NormalTok{D4_com<-}\KeywordTok{hclust}\NormalTok{(dist4,}\DataTypeTok{method=}\StringTok{"complete"}\NormalTok{)}
\CommentTok{#进行聚类分析,"single"为最短距离法,"complete"为最长距离法。}
\KeywordTok{plot}\NormalTok{(D4_single,}\DataTypeTok{hang=}\OperatorTok{-}\DecValTok{1}\NormalTok{,}\DataTypeTok{main=}\StringTok{"single"}\NormalTok{,}\DataTypeTok{sub=}\OtherTok{NULL}\NormalTok{,}\DataTypeTok{xlab=}\StringTok{"farm"}\NormalTok{)}
\end{Highlighting}
\end{Shaded}

\includegraphics{homework_files/figure-latex/unnamed-chunk-18-1.pdf}

\begin{Shaded}
\begin{Highlighting}[]
\KeywordTok{plot}\NormalTok{(D4_com,}\DataTypeTok{hang=}\OperatorTok{-}\DecValTok{1}\NormalTok{,}\DataTypeTok{main=}\StringTok{"complete"}\NormalTok{,}\DataTypeTok{sub=}\OtherTok{NULL}\NormalTok{,}\DataTypeTok{xlab=}\StringTok{"farm"}\NormalTok{)}
\end{Highlighting}
\end{Shaded}

\includegraphics{homework_files/figure-latex/unnamed-chunk-18-2.pdf}

\hypertarget{bux7528ux4e09ux4e2aux4e0dux540cux7684kux503cux5bf9ux519cux573aux4f5cux805aux7c7b}{%
\subsubsection{(b)用三个不同的K值对农场作聚类。}\label{bux7528ux4e09ux4e2aux4e0dux540cux7684kux503cux5bf9ux519cux573aux4f5cux805aux7c7b}}

取K=8,16,32进行K均值聚类分析如下

\begin{Shaded}
\begin{Highlighting}[]
\NormalTok{km1<-}\KeywordTok{kmeans}\NormalTok{(}\KeywordTok{scale}\NormalTok{(data4),}\DecValTok{8}\NormalTok{)}
\NormalTok{km1}\OperatorTok{$}\NormalTok{cluster  }\CommentTok{#输出分组结果}
\end{Highlighting}
\end{Shaded}

\begin{verbatim}
##  1  2  3  4  5  6  7  8  9 10 11 12 13 14 15 16 17 18 19 20 21 22 23 24 26 27 
##  3  4  8  8  4  3  3  3  1  3  8  8  8  5  7  8  5  8  6  8  8  5  1  5  8  8 
## 28 29 30 31 32 33 35 36 37 38 39 40 41 42 43 44 45 46 47 48 49 50 51 52 53 54 
##  5  5  2  8  5  5  8  1  8  7  8  8  3  1  8  1  4  8  8  5  8  5  8  2  8  5 
## 55 56 57 58 59 60 61 62 63 64 65 66 67 68 70 71 73 74 75 76 
##  8  7  7  1  5  1  1  4  8  3  1  6  8  3  5  7  3  6  3  4
\end{verbatim}

\begin{Shaded}
\begin{Highlighting}[]
\KeywordTok{print}\NormalTok{(}\KeywordTok{cutree}\NormalTok{(D4_single,}\DataTypeTok{k=}\DecValTok{8}\NormalTok{))  }\CommentTok{#输出最短距离法分K组的结果}
\end{Highlighting}
\end{Shaded}

\begin{verbatim}
##  1  2  3  4  5  6  7  8  9 10 11 12 13 14 15 16 17 18 19 20 21 22 23 24 26 27 
##  1  1  1  1  1  1  1  1  1  1  1  1  1  1  1  1  1  1  1  1  1  1  1  1  1  1 
## 28 29 30 31 32 33 35 36 37 38 39 40 41 42 43 44 45 46 47 48 49 50 51 52 53 54 
##  1  1  2  1  1  1  1  1  1  1  1  1  1  1  1  1  3  1  1  1  1  1  1  4  1  1 
## 55 56 57 58 59 60 61 62 63 64 65 66 67 68 70 71 73 74 75 76 
##  1  1  5  1  1  1  1  6  1  1  1  1  1  1  1  1  1  7  1  8
\end{verbatim}

\begin{Shaded}
\begin{Highlighting}[]
\NormalTok{km2<-}\KeywordTok{kmeans}\NormalTok{(}\KeywordTok{scale}\NormalTok{(data4),}\DecValTok{16}\NormalTok{)}
\NormalTok{km2}\OperatorTok{$}\NormalTok{cluster}
\end{Highlighting}
\end{Shaded}

\begin{verbatim}
##  1  2  3  4  5  6  7  8  9 10 11 12 13 14 15 16 17 18 19 20 21 22 23 24 26 27 
##  2  6 11 16  6  2  9  9 10 16 12 12 13 13  3 12  8 11 14 16 11  8  4  5 11 13 
## 28 29 30 31 32 33 35 36 37 38 39 40 41 42 43 44 45 46 47 48 49 50 51 52 53 54 
## 13  5 15 12  8 13 12  4  8  3 11 12  9  5 11 10  1 13 11  4 13 13 13 15 12  8 
## 55 56 57 58 59 60 61 62 63 64 65 66 67 68 70 71 73 74 75 76 
## 11  3  1  4  8  4 10  6 13  9  4 14 12  9  8  3  2 14  2  7
\end{verbatim}

\begin{Shaded}
\begin{Highlighting}[]
\KeywordTok{print}\NormalTok{(}\KeywordTok{cutree}\NormalTok{(D4_single,}\DataTypeTok{k=}\DecValTok{16}\NormalTok{))}
\end{Highlighting}
\end{Shaded}

\begin{verbatim}
##  1  2  3  4  5  6  7  8  9 10 11 12 13 14 15 16 17 18 19 20 21 22 23 24 26 27 
##  1  2  1  1  3  1  1  1  1  1  1  1  1  1  4  1  1  1  5  1  1  1  1  6  1  1 
## 28 29 30 31 32 33 35 36 37 38 39 40 41 42 43 44 45 46 47 48 49 50 51 52 53 54 
##  1  1  7  1  1  1  1  1  1  8  1  1  1  8  1  1  9  1  1  1  1  1  1 10  1  1 
## 55 56 57 58 59 60 61 62 63 64 65 66 67 68 70 71 73 74 75 76 
##  1 11 12  1  1  1 13 14  1  1  1  1  1  1  1  8  1 15  1 16
\end{verbatim}

\begin{Shaded}
\begin{Highlighting}[]
\NormalTok{km3<-}\KeywordTok{kmeans}\NormalTok{(}\KeywordTok{scale}\NormalTok{(data4),}\DecValTok{32}\NormalTok{)}
\NormalTok{km3}\OperatorTok{$}\NormalTok{cluster}
\end{Highlighting}
\end{Shaded}

\begin{verbatim}
##  1  2  3  4  5  6  7  8  9 10 11 12 13 14 15 16 17 18 19 20 21 22 23 24 26 27 
##  2 25 27 22 25  2 20 20  4 22 11 11 23 29 18 11 24 21 17 21 23  3  1 32 23 26 
## 28 29 30 31 32 33 35 36 37 38 39 40 41 42 43 44 45 46 47 48 49 50 51 52 53 54 
## 26 19 10  9 29 26  9  1  9 28 27 11 13 28  6  4 15 21 23  3 26 16 26 12 11  3 
## 55 56 57 58 59 60 61 62 63 64 65 66 67 68 70 71 73 74 75 76 
## 27 18  8  3 16  3  5 25 13 13  1 14 11 30 24  8  2  7  2 31
\end{verbatim}

\begin{Shaded}
\begin{Highlighting}[]
\KeywordTok{print}\NormalTok{(}\KeywordTok{cutree}\NormalTok{(D4_single,}\DataTypeTok{k=}\DecValTok{32}\NormalTok{))}
\end{Highlighting}
\end{Shaded}

\begin{verbatim}
##  1  2  3  4  5  6  7  8  9 10 11 12 13 14 15 16 17 18 19 20 21 22 23 24 26 27 
##  1  2  3  4  5  1  6  6  7  4  3  3  3  3  8  3  3  3  9  3  3  3 10 11  3  3 
## 28 29 30 31 32 33 35 36 37 38 39 40 41 42 43 44 45 46 47 48 49 50 51 52 53 54 
##  3 12 13  3  3  3  3 14  3 15  3  3  3 16  3 17 18  3  3  3  3  3  3 19  3  3 
## 55 56 57 58 59 60 61 62 63 64 65 66 67 68 70 71 73 74 75 76 
##  3 20 21 22  3 22 23 24  3  3 25 26  3 27  3 28 29 30 31 32
\end{verbatim}

将分类结果与(a)比较得当K=32时结果更好

\end{document}
